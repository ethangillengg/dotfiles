\documentclass[12pt]{article}

\usepackage [
	headsep=25pt,
	footnotesep=0pt,
	footskip=0pt,
]{geometry}
\usepackage{amsmath,amsthm,amssymb}
\usepackage{datetime}
\usepackage{multicol}
\usepackage{marvosym}
\usepackage{lastpage}
\usepackage{fancyhdr}
\usepackage{wasysym}
\usepackage{meta}
\usepackage{hyperref}
\usepackage[none]{hyphenat} % Prevent hyphenation
\usepackage[style=ieee]{biblatex}

% --------------------------------------------------------------
%                         Styles
% --------------------------------------------------------------
\addbibresource{hw.bib}
\urlstyle{same} % url same style as text
% --------------------------------------------------------------
%                         Environments
% --------------------------------------------------------------
\newenvironment{sol}[1][Solution]{\begin{trivlist}
		\item[\hskip \labelsep {\bfseries {#1}:}]}{\end{trivlist}}
% --------------------------------------------------------------
%                         Functions
% --------------------------------------------------------------
\newcommand{\Let}[2]{Let {$#1$} be #2.}
% --------------------------------------------------------------
%                         Header/Footer
% --------------------------------------------------------------
\pagestyle{fancy}
\fancyhead[L]{\doctitle}
\fancyhead[R]{page \thepage\ of \pageref*{LastPage}}
\fancyfoot{} % no footer
% --------------------------------------------------------------
%                         Title
% --------------------------------------------------------------
\begin{document}

\title{\doctitle}
\author{\ponename}
\date{\today}
\maketitle
\thispagestyle{empty} % remove page count for title page


\newpage
% --------------------------------------------------------------
%                         Assignment
% --------------------------------------------------------------
\setcounter{page}{1}

\bigskip

\section*{Exercise A: Moore's Law}
\subsection*{Question A}
According to the trend in device scaling historically observed by Moore’s Law, the number of transistors on a chip in 2025 should be how many times the number in 2015? \cite{hennessey2019computer}
\begin{sol}
	Moore's law states that the number of transistors should double every two years.
	\begin{itemize}
		\item \Let{Y}{number of years passed}
		\item \Let{n}{the coefficient of number of transistors}
	\end{itemize}
	\begin{align*}
		n & = 2\cdot{Y}    \\
		  & = 2(Y_1-Y_0)   \\
		  & = 2(2025-2015) \\
		  & = 20           \\
	\end{align*}
	According to Moore's law, the number of transistors in 2025 should be \textbf{20 times} the number in 2015.
\end{sol}

% --------------------------------------------------------------
%                         Bibliography
% --------------------------------------------------------------
\pagebreak
\printbibliography

% --------------------------------------------------------------
%                         The End
% --------------------------------------------------------------
\end{document}
