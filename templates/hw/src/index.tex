\newpage
\setcounter{page}{1}
\bigskip

% --------------------------------------------------------------
%                         Assignment
% --------------------------------------------------------------
\section*{Exercise A: Moore's Law}
\subsection*{Question A}
According to the trend in device scaling historically observed by Moore’s Law, the number of transistors on a chip in 2025 should be how many times the number in 2015? \cite{hennessey2019computer}
\begin{sol}
	Moore's law states that the number of transistors should double every two years.
	\begin{itemize}
		\item \Let{Y}{number of years passed}
		\item \Let{n}{the coefficient of number of transistors}
	\end{itemize}
	\begin{align*}
		n & = 2\cdot{Y}    \\
		  & = 2(Y_1-Y_0)   \\
		  & = 2(2025-2015) \\
		  & = 20           \\
	\end{align*}
	According to Moore's law, the number of transistors in 2025 should be \textbf{20 times} the number in 2015.
\end{sol}


% --------------------------------------------------------------
%                         Bibliography
% --------------------------------------------------------------
% \pagebreak
% \printbibliography
